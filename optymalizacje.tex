
\subsection{Ogólnie optymalizacje}

\begin{itemize}
    \item \textsc{Constant folding/propagation}

\begin{figure}[H]
 \begin{minipage}{0.1\textwidth}
  \centering
  \begin{minted}{python}
t1 := 7
t2 := t1 - 1
t3 := t2 * t2
a := b + t3
  \end{minted}
 \end{minipage}
 \begin{minipage}{0.04\textwidth}
 $\rightarrow$
 \end{minipage}
 \begin{minipage}{0.2\textwidth}
  \centering
  \begin{minted}{python}
a := b + 36
  \end{minted}
 \end{minipage}
\end{figure}

\begin{figure}[H]
 \begin{minipage}{0.2\textwidth}
  \centering
  \begin{minted}{python}
r1 := 1
...
r5 = phi(entry: r1, L5: r7) 
  \end{minted}
 \end{minipage}
 \begin{minipage}{0.04\textwidth}
 $\rightarrow$
 \end{minipage}
 \begin{minipage}{0.2\textwidth}
  \centering
  \begin{minted}{python}
r1 := 1
...
r5 = phi(entry: 1, L5: r7) 
  \end{minted}
 \end{minipage}
\end{figure}

\item \textsc{Copy propagation}

\begin{figure}[H]
 \begin{minipage}{0.1\textwidth}
  \centering
  \begin{minted}{python}
x := y
t := x + 1
  \end{minted}
 \end{minipage}
 \begin{minipage}{0.04\textwidth}
 $\rightarrow$
 \end{minipage}
 \begin{minipage}{0.2\textwidth}
  \centering
  \begin{minted}{python}
x := y
t := y + 1
  \end{minted}
 \end{minipage}
\end{figure}

\item \textsc{Local Common Subexpression Elimination}

\begin{figure}[H]
 \begin{minipage}{0.1\textwidth}
  \centering
  \begin{minted}{python}
t6 := 4*i
x := a[t6]
t7 := 4*i
t8 := 4*j
t9 := a[t8]
a[t7] := t9
t10 := 4*j
a[t10] := x
goto L5
  \end{minted}
 \end{minipage}
 \begin{minipage}{0.04\textwidth}
 $\rightarrow$
 \end{minipage}
 \begin{minipage}{0.2\textwidth}
  \centering
  \begin{minted}{python}
t6 := 4*i
x := a[t6]

t8 := 4*j
t9 := a[t8]
a[t6] := t9

a[t8] := x
goto L5
  \end{minted}
 \end{minipage}
\end{figure}

\item \textsc{Global Common Subexpression Elimination}

To samo, ale między blokami prostymi

\item \textsc{Dead Code Elimination}

Wywalamy nieużywane rejestry

\item \textsc{Moving code out of the loop}

\begin{figure}[H]
 \begin{minipage}{0.15\textwidth}
  \centering
  \begin{minted}{c}
  |$\ $|
while(i<=n-3) {
    s += a[i];
    i++;
}

  \end{minted}
 \end{minipage}
 \begin{minipage}{0.04\textwidth}
 $\rightarrow$
 \end{minipage}
 \begin{minipage}{0.2\textwidth}
  \centering
  \begin{minted}{python}
t = n-3;
while(i<=t) {
    s += a[i];
    i++;
}
  \end{minted}
 \end{minipage}
\end{figure}

\item \textsc{Strength reduction} - zastępowanie drogich operacji prostszymi (np. mnożenie dodawaniem)

\begin{figure}[H]
 \begin{minipage}{0.15\textwidth}
  \centering
  \begin{minted}{c}
    i := 0;
goto L2
L1: i := i+1
    t2 := 4*i
    t3 := a[t2]
    s := s + t3
L2: if(a[t2]<=k) goto L1

  \end{minted}
 \end{minipage}
 \begin{minipage}{0.04\textwidth}
 $\rightarrow$
 \end{minipage}
 \begin{minipage}{0.2\textwidth}
  \centering
  \begin{minted}{python}
    t2 := 0;
    goto L2
L1: t2 := t2 + 4 
    t3 := a[t2]
    s := s + t3
L2: if(a[t2]<=k) goto L1

  \end{minted}
 \end{minipage}
\end{figure}

\item \textsc{Loop unrolling, loop inlining}

Loop unrolling - zmniejszenie liczby obrotów poprzez wykonywanie kilku obrotów pętli na raz (np. na x86 zamiast robić strcmp na bajtach - robić na całych słowach)

Loop inlining - unrolling do zera

\end{itemize}

\subsection{Przykład:}


\begin{minted}{c}
for (m = a[i]; (i > 0) && (3 + m < a[i-]); i--)
    a[i] = a[i-1];
a[i] = m;
        czwórkowy:                      SSA
 1. L0:                             L0:
 2.    m = a[i]                       m = a[i]
 3. L1:                             L1:
 4.     if i > 0 goto L2 else L3       i1 = |$\phi$|(L0: i, L4: i2)
 5.                                    if i1 > 0 goto L2 else L3
 6. l2:                             L2:
 7.     t1 = 3 + m                     t1 = 3 + m
 8.     t2 = i - 1                     t2 = i1 - 1
 9.     t3 = a[t2]                     t3 = a[t2]
10.     if t1 < t3 goto L4 else L3     if t1 < t3 goto L4 else L3
11. L4:                             L4:
12.     t4 = i - 1                     t4 = i1 - 1
13.     t5 = a[t4]                     t5 = a[t4]
14.     a[i] = t5                      a[i1] = t5
15.     i = i - 1                      i2 = i1 - 1
16.     goto L1                        goto L1
17. L3:                             L3:
18.      a[i] = m                      a[i1] = m
\end{minted}

\begin{enumerate}
    \item const folding: stałe wyliczamy w czasie kompilacji
    \item common subexpression substitution: tam gdzie występuje to samo, redukujemy (ale tylko obliczenia), np. w linijce 12 i 15 wyliczamy $i1 - 1$, które zostało już policzone w linijce 8. Uwaga na tablice!
    
    \item copy propagation:
    
    widzimy $y$ po prawej stronie równości
    
    szukamy $y = a$ (gdzie a jest po prostu rejestrem)
    
    zamieniamy $y$ na $a$
    
    \begin{minted}{c}
                     krok 2   krok 3       krok 2     krok 3
 1. L0:
 2.   m = a[i]
 3. L1:
 4.    i1 = |$\phi$|(L0: i, L4: i2)
 5.    if i1 > 0 
        goto L2 else L3
 6. L2:
 7.    t1 = 3 + m
 8.    t2 = i1 - 1
 9.    t3 = a[t2]
10.    if t1 < t3
        goto L4 else L3
11. L4:
12.    t4 = i1 - 1    t4 = t2
13.    t5 = a[t4]              t5 = a[t2]   t5 = t3
14.    a[i1] = t5                                     a[i1] = t3
15.    i2 = i1 - 1    i2 = t2
16.    goto L1
17. L3:
18.    a[i1] = m    
    \end{minted}
    
    
    Teraz znowu powtarzamy te kroki aż nic się nie będzie zmieniało

    \item Dead code optimization i wywalamy 12, 13 linijkę
    \item Wyciąganie stałych wyrażeń przed pętlę
        \begin{minted}{c}
 1. L0:
 2.   m = a[i]
 3.   t1 = 3 + m     <- tutaj
 4. L1:
 5.    i1 = |$\phi$|(L0: i, L4: i2)   // <- phi(L0: i, L4: t2)
 6.    if i1 > 0 goto L2 else L3
 7. L2:
 8.    t2 = i1 - 1
 9.    t3 = a[t2]
10.    if t1 < t3 goto L4 else L3
11. L4:
12.    a[i1] = t3
13.    i2 = t2               // <- usuwamy
14.    goto L1
15. L3:
16.    a[i1] = m    
    \end{minted}
    
    Teraz możemy znowu rozpropagowac kopię i w linijce 5. użyć $t2$ zamiast $i2$ (a następnie w ramach usuwania martwego kodu usunąć przypisanie do $i2$)
    
    
\end{enumerate}

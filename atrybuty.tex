Do symboli (nieterminali i terminali) w gramatyce dodajemy \textsc{atrybuty}, tj. informację, jaką ze sobą ten symbol niesie.

\begin{itemize}
    \item atrybuty syntezowane
    
    gdy mamy produkcję $X \ra \alpha$ i ustawiamy atrybut nieterminalowi po lewej stronie - X. Ten atrybut będzie propagowany w górę. 
    
    \item atrybuty dziedziczone
    
    gdy mamy produkcję $X \ra \alpha Y \beta$ i ustawiamy atrybut nieterminalowi po prawej stronie - Y. Ten atrybut będzie propagowany w dół (gdy będziemy parsować $Y \ra \gamma$, ten nieterminal może się dowiedzieć co przyszło do niego z góry)
    
    \item atrybuty wbudowane - atrybuty nadawane terminalom (identyczne zasady)
\end{itemize}

\textsc{Przykład}:

$\quad Z \ra S$

$\quad S \ra X | XS$

$\quad X \ra ab | aXb$

Chcemy nadać atrybut $Z.ak$ na true wtw gdy słowo $w$ jest postaci $(a^nb^n)^n$

$\quad Z \ra S$   \{$Z.ok = S.ileX == S.\#ab\ and\ S.ok$\}

$\quad S \ra X$    \{$S.ileX = 1$, $S.\#ab = X_0.ile$, $S.ok = true$\}

$\quad S_0 \ra XS_1$ \{$S_0.ileX = 1 + S_1.ileX$, $S_0.ok = S_1.ok\ and\ X.ile == S1.\#ab$, $S_0.\#ab = S_1.\#ab$\}

$\quad X \ra ab$ \{$X.ile = 1$\}

$\quad X_0 \ra aX_1b$ \{$X_0.ile = 1 + X_1.ile$\}

$X.ile$ - ile wynosi $k$ w wyprodukowanym $a^kb^k$

$S.ileX$ - ile Xów wyprodukowało S

$S.ok$ - czy wszystkie X które ten S wyprowadził, $X.ile$ ma tę samą wartość

$S.\#ab$ - jaką wartość ma $X.ile$, którą wyprowadził (bo każdy X ma mieć tę samą)
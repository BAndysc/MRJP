\subsection{x86}
\begin{minted}{c}
mov EAX, EBX             EAX = EBX
lea EAX, [4 * EBX]       EAX = 4 * EBX
add EAX, [EBP - 8]       EAX = EAX + wartość pod (EBP - 8)
cmp EAX, EBX             
jg label                 skok gdy EAX > EBX
jl label                 skok gdy EAX < EBX
jge label                skok gdy EAX >= EBX
jle label                skok gdy EAX <= EBX
je/jne label             
push ebp                 standardowy prolog
mov rbp, esp             standardowy prolog
leave                    standardowy epilog
ret                      standardowy epilog
imul EBX                 EDX:EAX = EAX * EBX
imul EBX, ECX            EDX:EAX = EBX * ECX
idiv ECX                 EAX = EDX:EAX / ECX, EDX = EDX:EAX % ECX
mul                      mnożenie bez znaku
div                      dzielenie bez znaku
not EAX                  EAX = ~EAX
neg EAX                  EAX = -EAX
\end{minted}

\subsubsection{Protokół wywołania GCC+libc (aka “cdecl”)}

\begin{itemize}
    \item rejestry: $EAX$, $ECX$, $EDX$, $EBX$, $ESP$, $EBP$, $ESI$, $EDI$
    \item $EAX$, $ECX$, $EDX$ - caller saved (== można je zmieniać w funkcji bez obaw)
    \item pozostałe - callee saved (nie można zmieniać bez przywrócenia starej wartości przed wyjściem)
    \item argumenty od końca (najpierw wkładamy ostatnie)
    \item wynik w $EAX$
    \item gcc wyrównuje stos do 16 bajtów, tzn. przed zrobieniem $call$ stos jest wyrównany, tzn. po $call$, $ESP \equiv 4  mod 16$, tzn. po $push\ ebp$: $ESP \equiv 8 mod 16$.
    \item ale slajdy mówią o wyrównaniu tylko w przypadku x86\_64
\end{itemize}

\subsubsection{x86\_64}
\begin{itemize}
    \item rejestry: $RAX$, $RBX$, $RCX$, $RDX$, $RBP$, $RSP$, $RSI$, $RDI$, $R8$, $R9$, $R10$, $R11$, $R12$, $R13$, $R14$, $R15$
    \item inty przekazywane w: $EDI$, $ESI$, $EDX$, $ECX$, $R8D$, $R9D$
    \item $RBP$, $RBX$ and $R12$ - $R15$ callee saved
    \item stos na pewno wyrównany
\end{itemize}


\subsection{JVM}
\begin{minted}{c}
iconst_0            push 0
istore_2            pop from stack to 2. local var
iload_3             push 3. local variable
if_icmpge           skacze gdy wartość głębsza jest >= wartości na topie
aload_0             załaduj tablicę ze zmiennej lokalnej 0
iaload              wczytaj wartość z tablicy (głębiej tablica)
iinc X Y            zwiększ zmienną lokalną X o Y
dup                 duplikuj ostatnią rzecz na stosie
dup2                duplikuje dwie rzeczy ze stosu
ifeq label          skacze gdy wartość na stosie == 0
iastore             zapisuje do tablicy (od góry: wartość, indeks, tablica)
bipush 123          umieszcza bajt na stosie
idiv, iadd, isub, imul
\end{minted}
